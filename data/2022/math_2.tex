\documentclass[10pt]{article}
\usepackage[utf8]{inputenc}
\usepackage[T1]{fontenc}
\usepackage{amsmath}
\usepackage{amsfonts}
\usepackage{amssymb}
\usepackage[version=4]{mhchem}
\usepackage{stmaryrd}

\begin{document}
\section{MOTIOON JEE Advanced}
\section{SECTION 1 (Maximum marks: 24)}
\begin{itemize}
  \item This section contains EIGHT (08) questions.

  \item The answer to each question is a SINGLE DIGIT INTEGER ranging from 0 TO 9, BOTH INCLUSIVE.

  \item For each question, enter the correct integer corresponding to the answer using the mouse and the on screen virtual numeric keypad in the place designated to enter the answer.

  \item Answer to each question will be evaluated according to the following marking scheme:

\end{itemize}

Full Marks : + 3 If ONLY the correct integer is entered;

Zero Marks : 0 If the question is unanswered;

Negative Marks : -1 In all other cases.

\begin{enumerate}
  \item Let $\alpha$ and $\beta$ be real numbers such that $-\frac{\pi}{4}<\beta<0<\alpha<\frac{\pi}{4}$. If $\sin (\alpha+\beta)=\frac{1}{3}$ and $\cos (\alpha-\beta)=$
\end{enumerate}

Ans. 1 $\frac{2}{3}$, then the greatest integer less than or equal to $\left(\frac{\sin \alpha}{\cos \beta}+\frac{\cos \beta}{\sin \alpha}+\frac{\cos \alpha}{\sin \beta}+\frac{\sin \beta}{\cos \alpha}\right)^{2}$ is

\begin{enumerate}
  \setcounter{enumi}{1}
  \item If $y(x)$ is the solution of the differential equation
\end{enumerate}

$x d y-\left(y^{2}-4 y\right) d x=0$ for $x>0, \quad y(1)=2$,

Ans. 8 and the slope of the curve $y=y(x)$ is never zero, then the value of $10 y(\sqrt{2})$

\begin{enumerate}
  \setcounter{enumi}{2}
  \item The greatest integer less than or equal to
\end{enumerate}

$$
\int_{1}^{2} \log _{2}\left(x^{3}+1\right) d x+\int_{1}^{\log _{2} 9}\left(2^{x}-1\right)^{\frac{1}{3}} d x \text { is }
$$

Ans. 5

\begin{enumerate}
  \setcounter{enumi}{3}
  \item The product of all positive real values of $x$ satisfying the equation $x^{\left(16\left(\log _{5} x\right)^{3}-68 \log _{5} x\right)}=5^{-16}$ is
\end{enumerate}

Ans. 1

\begin{enumerate}
  \setcounter{enumi}{4}
  \item If $\beta=\lim _{x \rightarrow 0} \frac{e^{x^{3}}-\left(1-x^{3}\right)^{\frac{1}{3}}+\left(\left(1-x^{2}\right)^{\frac{1}{2}}-1\right) \sin x}{x \sin ^{2} x}$ then the value of $6 \beta$ is
\end{enumerate}

Ans. 5

\begin{enumerate}
  \setcounter{enumi}{5}
  \item Let $\beta$ be a real number. Consider the matrix
\end{enumerate}

$$
A=\left(\begin{array}{ccc}
\beta & 0 & 1 \\
2 & 2 & -2 \\
3 & 1 & -2
\end{array}\right)
$$

Ans. 3 If $A^{7}-(\beta-1) A^{6}-\beta A^{5}$ is a singular matrix, then the value of $9 \beta$ is

\section{MOTIOON JEE Advanced}
Question Paper

with Answer

\begin{enumerate}
  \setcounter{enumi}{6}
  \item Consider the hyperbola
\end{enumerate}

$\frac{x^{2}}{100}-\frac{y^{2}}{64}=1$

with foci at $S$ and $S_{1}$, where $S$ lies on the positive $x$-axis. Let $P$ be a point on the hyperbola, in the first quadrant. Let $\angle S P S_{1}=\alpha$, with $\alpha<\frac{\pi}{2}$. The straight line passing through the point $S$ and having the same slope as that of the tangent at $P$ to the hyperbola, intersects the straight line $S_{1} P$ at $\mathrm{P}_{1}$. Let $\delta$ be the distance of $\mathrm{P}$ from the straight line $\mathrm{SP}_{1}$, and $=S_{1} P$. Then the greatest integer less than or equal to $\frac{\beta \delta}{9} \sin \frac{\alpha}{2}$ is

Ans. 7

\begin{enumerate}
  \setcounter{enumi}{7}
  \item Consider the functions $\mathrm{f}, \mathrm{g}: \mathrm{R} \rightarrow \mathrm{R}$ defined by
\end{enumerate}

$f(x)=x^{2}+\frac{5}{12}$ and $g(x)=\left\{\begin{array}{cl}2\left(1-4 \frac{|x|}{3}\right), & |x| \leq \frac{3}{4}, \\ 0, & |x|>\frac{3}{4} .\end{array}\right.$. If $\alpha$ is the area of the region $\left\{(x, y) \in R \times R:|x| \leq \frac{3}{4}, 0 \leq y \leq \min \{f(x), g(x)\}\right\}$, then the value of $9 \alpha$ is

Ans. 6

\section{Section 2 (Maximum marks : 24)}
\begin{itemize}
  \item This section contains Six (06) question.

  \item Each question has Four option (A), (B), (C) and (D). ONE OR MORE THAN ONE of these four options(s)(are) correct answer (s).

  \item For each question, choose the option(s) corresponding to (all ) the correct answer(s).

  \item Answer to each question will be evaluated according to the following marking scheme:

  \item Full Marks : $\quad+4$ ONLY if (all) the correct option(s) is (are) chosen;

  \item Partial Marks : +3 If all the four options are correct but ONLY three options are chosen;

  \item Partial Marks : +2 If three or more options are correct by ONLY two options are chosen, both of which are correct;

  \item Partial Marks : + 1 If two or more options are correct but ONLY option is chosen and it is a correct option;

  \item Zero Marks : 0 is unanswered;

  \item Negative Marks : -2 In all other cases.

\end{itemize}

\begin{enumerate}
  \setcounter{enumi}{8}
  \item Let PQRS be quadrilateral in a plane, where $\mathrm{QR}=1, \angle P Q R=\angle Q R S=70^{\circ}, \angle P Q S=15^{\circ}$ and $\angle P R S=40^{\circ}$. If $\angle R P S=\theta^{\circ}, P Q=\alpha$ and $P S=\beta$, then the interval(s) that contains(s) the value of $4 \alpha \beta \sin \theta^{\circ}$ is/are
(A) $(0, \sqrt{2})$
(B) $(1,2)$
(C) $(\sqrt{2}, 3)$
(D) $(2 \sqrt{2}, 3 \sqrt{2})$
\end{enumerate}

Ans. A,B

\section{MOTIOON JEE Advanced}
\begin{enumerate}
  \setcounter{enumi}{9}
  \item Let
\end{enumerate}

$\alpha=\sum_{k=1}^{\infty} \sin ^{2 k}\left(\frac{\pi}{6}\right)$.

Let $g:[0,1] \rightarrow R$ be the function defined by

$g(x)=2^{\alpha x}+2^{\alpha(1-x)}$

Then, which of the following statements is/are TRUE?
(A) The minimum value of $g(x)$ is $2^{\frac{7}{6}}$
(B) The maximum value of $g(x)$ is $1+2^{\frac{1}{3}}$
(C) The function $g(x)$ attains its maximum at more than one point
(D) The function $\mathrm{g}(\mathrm{x})$ attains its minimum at more than one point

Ans. A,B,C

\begin{enumerate}
  \setcounter{enumi}{10}
  \item Let $\bar{z}$ denote the complex conjugate of a complex number $\mathrm{z}$. If $\mathrm{z}$ is a non-zero complex number for which both real and imaginary parts of
\end{enumerate}

$$
(\bar{z})^{2}+\frac{1}{z^{2}}
$$

are integers, then which of the following is/are possible value(s) of $|z|$ ?
(A) $\left(\frac{43+3 \sqrt{205}}{2}\right)^{\frac{1}{4}}$
(B) $\left(\frac{7+\sqrt{33}}{4}\right)^{\frac{1}{4}}$
(C) $\left(\frac{9+\sqrt{65}}{4}\right)^{\frac{1}{4}}$
(D) $\left(\frac{7+\sqrt{13}}{6}\right)^{\frac{1}{4}}$

Ans. A

\begin{enumerate}
  \setcounter{enumi}{11}
  \item Let $\mathrm{G}$ be a circle of radius $\mathrm{R}>0$. Let $\mathrm{G}_{1}, G_{2}, \ldots . ., \mathrm{G}_{\mathrm{n}}$ be $\mathrm{n}$ circles of equal radius $r>0$. Suppose each of the $n$ circles $G_{1}, G_{2}, \ldots . ., G_{n}$ touches the circle $G$ externally. Also, for $\mathrm{i}=1,2, \ldots . ., \mathrm{n}-1$, the circle $G_{i}$ touches $G_{i+1}$ externally, and $G_{n}$ touches $G_{1}$ externally. Then, which of the following statements is/are TRUE?
(A) If $\mathrm{n}=4$, then $(\sqrt{2}-1) r<R$
(B) If $n=5$, then $r<R$
(C) If $\mathrm{n}=8$, then $(\sqrt{2}-1) r<R$
(D) If $\mathrm{n}=12$, then $\sqrt{2}(\sqrt{3}+1) r>R$
\end{enumerate}

Ans. C,D

\begin{enumerate}
  \setcounter{enumi}{12}
  \item Let $\hat{\imath}, \hat{\jmath}$ and $\hat{k}$ be the unit vectors along the three positive coordinate axes. Let
\end{enumerate}

$$
\begin{array}{rlr}
\vec{a} & =3 \hat{\imath}+\hat{\jmath}-\hat{k} & \\
\vec{b} & =\hat{\imath}+b_{2} \hat{\jmath}+b_{3} \hat{k}, & b_{2} b_{3} \in R \\
\vec{c} & =c_{1} \hat{\imath}+c_{2} \hat{\jmath}+c_{3} \hat{k}, & c_{1}, c_{2}, c_{3} \in R
\end{array}
$$

be three vectors such that $\mathrm{b}_{1} \mathrm{~b}_{3}>0, \vec{a} \cdot \vec{b}=0$ and

$$
\left(\begin{array}{ccc}
0 & -c_{3} & c_{2} \\
c_{3} & 0 & -c_{1} \\
-c_{2} & c_{1} & 0
\end{array}\right)\left(\begin{array}{c}
1 \\
b_{2} \\
b_{3}
\end{array}\right)=\left(\begin{array}{c}
3-c_{1} \\
1-c_{2} \\
-1-c_{3}
\end{array}\right)
$$

Then, which of the following is/are True?
(A) $\vec{a} \cdot \vec{c}=0$
(B) $\vec{b} \cdot \vec{c}=0$
(C) $|\vec{b}|>\sqrt{10}$
(D) $|\vec{c}| \leq \sqrt{11}$

Ans. B,C,D

\section{Motíon JEE Advanced}
Question Paper

with Answer

\begin{enumerate}
  \setcounter{enumi}{13}
  \item For $x \in R$, let the function $\mathrm{y}(\mathrm{x})$ be the solution of the differential equation $\frac{d y}{d x}+12 y=\cos \left(\frac{\pi}{12} x\right), y(0)=0$
\end{enumerate}

Then, which of the following statements is/are TRUE?

(A) $\mathrm{y}(\mathrm{x})$ is an increasing function

(B) $y(x)$ is a decreasing function

(C) There exists a real number $\beta$ such that the line $y=\beta$ intersects the curve $y=y(x)$ at infinitely many points

Ans. $\mathrm{C}$

(D) $\mathrm{y}(\mathrm{x})$ is a periodic function

\begin{itemize}
  \item This section contains FOUR (04) questions.
\end{itemize}

\section{SECTION 3 (Maximum marks: 12)}
\begin{itemize}
  \item Each question has FOUR options (A), (B), (C) and (D). ONLY ONE of these four options is the correct answer.

  \item For each question, choose the option corresponding to the correct answer.

  \item Answer to each question will be evaluated according to the following marking scheme:

\end{itemize}

Full Marks : : +3 If ONLY the correct option is chosen;

Zero Marks : 0 If none of the options is chosen (i.e. the question is unanswered);

Negative Marks : :-1 In all other cases.

\begin{enumerate}
  \setcounter{enumi}{14}
  \item Consider 4 boxes, where each box contains 3 red balls and 2 blue balls. Assume that all 20 balls are distinct. In how many different ways can 10 balls be chosen from these 4 boxes so that from each box at least one red ball and one blue ball are chosen?
\end{enumerate}

Ans. $\quad \mathbf{A}$
(A) 21816
(B) 85536
(C) 12096
(D) 156816

\begin{enumerate}
  \setcounter{enumi}{15}
  \item If $M=\left(\begin{array}{cc}\frac{5}{2} & \frac{3}{2} \\ -\frac{3}{2} & -\frac{1}{2}\end{array}\right)$, then which of the following matrices is equal $M^{2022}$ ?
\end{enumerate}

Ans. (A)
(A) $\left(\begin{array}{cc}3034 & 3033 \\ -3033 & -3032\end{array}\right)$
(B) $\left(\begin{array}{ll}3034 & -3033 \\ 3033 & -3032\end{array}\right)$
(C) $\left(\begin{array}{cc}3033 & 3032 \\ -3032 & -3031\end{array}\right)$
(D) $\left(\begin{array}{cc}3032 & 3031 \\ -3031 & -3030\end{array}\right)$

\begin{enumerate}
  \setcounter{enumi}{16}
  \item Suppose that
\end{enumerate}

Box-I contains 8 red, 3 blue and 5 green balls,

Box-II contains 24 red, 9 blue and 15 green balls,

Box-III contains 1 blue, 12 green and 3 yellow balls,

Box-IV contains 10 green, 16 orange and 6 white balls.

A ball is chosen randomly from Box-I; call this ball $b$. If $b$ is red then a ball is chosen randomly from Box-II, if $b$ is blue then a ball is chosen randomly from Box-III, and if $b$ is green then a ballis chosen randomly from Box-IV. The conditional probability of the event 'one of the chosen balls is white' given that the event 'at least one of the chosen balls is green' has happened, is equal to
(A) $\frac{15}{256}$
(B) $\frac{3}{16}$
(C) $\frac{5}{52}$
(D) $\frac{1}{8}$

Ans. $\mathrm{C}$

JEE V-STAR BATCH

\section{MOTÍON JEE Advanced}
\begin{enumerate}
  \setcounter{enumi}{17}
  \item For positive integer $n$, define
\end{enumerate}

$f(n)=n+\frac{16+5 n-3 n^{2}}{4 n+3 n^{2}}+\frac{32+n-3 n^{2}}{8 n+3 n^{2}}+\frac{48-3 n-3 n^{2}}{12 n+3 n^{2}}+\ldots .+\frac{25 n-7 n^{2}}{7 n^{2}}$.

Then, the value of $\lim _{n \rightarrow \infty} f(n)$ is equal to

Ans. B
(A) $3+\frac{4}{3} \log _{e} 7$
(B) $4-\frac{3}{4} \log _{e}\left(\frac{7}{3}\right)$
(C) $4-\frac{4}{3} \log _{e}\left(\frac{7}{3}\right)$
(D) $3+\frac{3}{4} \log _{e} 7$


\end{document}