\documentclass[10pt]{article}
\usepackage[utf8]{inputenc}
\usepackage[T1]{fontenc}
\usepackage{amsmath}
\usepackage{amsfonts}
\usepackage{amssymb}
\usepackage[version=4]{mhchem}
\usepackage{stmaryrd}
\usepackage{bbold}

\begin{document}
\section{MOTIOON JEE Advanced}
Question Paper

with Answer

\section{SECTION 1 (Maximum Marks: 24)}
\begin{itemize}
  \item This section contains EIGHT (08) questions.

  \item The answer to each question is a NUMERICAL VALUE.

  \item For each question, enter the correct numerical value of the answer using the mouse and the onscreen virtual numeric keypad in the place designated to enter the answer. If the numerical value has more than two decimal places, truncate/roundoff the value to TWO decimal places.

  \item Answer to each question will be evaluated according to the following marking scheme:

\end{itemize}

Full Marks : +3 ONLY if the correct numerical value is entered;

Zero Marks : 0 In all other cases.

\begin{enumerate}
  \item Considering only the principal values of the inverse trigonometric functions, the value of $\frac{3}{2} \cos ^{-1} \sqrt{\frac{2}{2+\pi^{2}}}+\frac{1}{4} \sin ^{-1} \frac{2 \sqrt{2} \pi}{2+\pi^{2}}+\tan ^{-1} \frac{\sqrt{2}}{\pi}$ is
\end{enumerate}

Ans. 2.36

\begin{enumerate}
  \setcounter{enumi}{1}
  \item Let $\alpha$ be a positive real number. Let $f: \mathbb{R} \rightarrow \mathbb{R}$ and $g:(\alpha, \infty) \rightarrow \mathbb{R}$ be the functions defined by
\end{enumerate}

$$
f(x)=\sin \left(\frac{\pi x}{12}\right) \text { and } g(x)=\frac{2 \log _{e}(\sqrt{x}-\sqrt{\alpha})}{\log _{e}\left(e^{\sqrt{x}}-e^{\sqrt{\alpha}}\right)}
$$

Then the value of $\lim _{x \rightarrow \alpha^{+}} f(g(x))$ is

Ans. 0.5

\begin{enumerate}
  \setcounter{enumi}{2}
  \item In a study about a pandemic, data of 900 persons was collected. It was found that
\end{enumerate}

190 persons had symptom of fever,

220 persons had symptom of cough,

220 persons had symptom of breathing problem,

330 persons had symptom of fever or cough or both,

350 persons had symptom of cough or breathing problem or both,

340 persons had symptom of fever or breathing problem or both,

30 persons had all three symptoms (fever, cough and breathing problem).

If a person is chosen randomly from these 900 persons, then the probability that the person has at most one symptom is

Ans. $\quad 0.80$

\begin{enumerate}
  \setcounter{enumi}{3}
  \item Let $z$ be a complex number with non-zero imaginary part. If $\frac{2+3 z+4 z^{2}}{2-3 z+4 z^{2}}$ is a real number, then the value of $|z|^{2}$ is
\end{enumerate}

Ans. 0.5

\begin{enumerate}
  \setcounter{enumi}{4}
  \item Let $\bar{z}$ denote the complex conjugate of a complex number $z$ and let $i=\sqrt{-1}$. In the set of complex numbers, the number of distinct roots of the equation $\bar{z}-z^{2}=i\left(\bar{z}+z^{2}\right)$ is
\end{enumerate}

Ans. 4

\section{MOTIOON JEE Advanced}
Question Paper

with Answer

\begin{enumerate}
  \setcounter{enumi}{5}
  \item Let $l_{1}, l_{2}, \ldots, l_{100}$ be consecutive terms of an arithmetic progression with common difference $d 1$, and let $w_{1}, w_{2}, \ldots, w_{100}$ be consecutive terms of another arithmetic progression with common difference $\mathrm{d}_{2}$, where $\mathrm{d}_{1} \mathrm{~d}_{2}=10$. For each $i=1,2, \ldots, 100$, let $\mathrm{R}_{\mathrm{i}}$ be a rectangle with length $l_{\mathrm{i}}$, width $W_{i}$ and area $A_{i}$. If $A_{51}-A_{50}=1000$, then the value of $A_{100}-A_{90}$ is
\end{enumerate}

Ans. 18900

\begin{enumerate}
  \setcounter{enumi}{6}
  \item The number of 4-digit integers in the closed interval [2022, 4482] formed by using the digits $0,2,3,4,6,7$ is
\end{enumerate}

Ans. 569

\begin{enumerate}
  \setcounter{enumi}{7}
  \item Let $A B C$ be the triangle with $A B=1, A C=3$ and $\angle B A C=\frac{\pi}{2}$. If a circle of radius $r>0$ touches the sides $A B, A C$ and also touches internally the circumcircle of the triangle $A B C$, then the value of $r$ is
\end{enumerate}

Ans. $\quad 0.84$

\section{SECTION B}
\begin{itemize}
  \item This section contains SIX (06) questions.

  \item Each question has FOUR options (A), (B), (C) and (D). ONE OR MORE THAN ONE of these four option(s) is(are) correct answer(s).

  \item For each question, choose the option(s) corresponding to (all) the correct answer(s).

  \item Answer to each question will be evaluated according to the following marking scheme:

\end{itemize}

Full Marks

Partial Marks

Partial Marks

Partial Marks

Zero Marks : +4 ONLY if (all) the correct option(s) is(are) chosen;

$:+3$ If all the four options are correct but ONLY three options are chosen;

: +2 If three or more options are correct but ONLY two options are chosen, both of which are correct;

:+1 If two or more options are correct but ONLY one option is chosen and it is a correct option;

: 0 If none of the options is chosen (i.e. the question is unanswered);

-2 In all other cases.

\begin{enumerate}
  \setcounter{enumi}{8}
  \item Consider the equation
\end{enumerate}

Ans. C,D

$$
\int_{1}^{e} \frac{\left(\log _{e} x\right)^{\frac{1}{2}}}{x\left(a-\left(\log _{e} x\right)^{\frac{3}{2}}\right)^{2}} d x=1, \quad a \in(-\infty, 0) \cup(1, \infty)
$$

Which of the following statements is/are TRUE?
(A) No a satisfies the above equation
(B) An integer a satisfies the above equation
(C) An irrational number a satisfies the above equation
(D) More than one a satisfies the above equation

\section{MOTIOON JEE Advanced}
Question Paper

with Answer

\begin{enumerate}
  \setcounter{enumi}{9}
  \item Let $a_{1}, a_{2}, a_{3}, \ldots$ beanarithmetic progression with $a_{1}=7$ and common difference8. Let $T_{1}, T_{2}, T_{3}, \ldots$. be such that $T_{1}=3$ and $T_{n+1}-T_{n}=a_{n}$ for $n \geq 1$. Then, which of the following is/are TRUE $?$
\end{enumerate}

Ans. $B, C$
(A) $\mathrm{T}_{20}=1604$
(C) $T_{30}=3454$
(B) $\sum_{k=1}^{20} T_{K}=10510$
(D) $\sum_{k=1}^{30} T_{K}=35610$

\begin{enumerate}
  \setcounter{enumi}{10}
  \item Let $\mathrm{P}_{1}$ and $\mathrm{P}_{2}$ be two planes given by
\end{enumerate}

$$
\begin{aligned}
& P_{1}: 10 x+15 y+12 z-60=0 \\
& P_{2}:-2 x+5 y+4 z-20=0 .
\end{aligned}
$$

Which of the following straight lines can be an edge of some tetrahedron whose two faces lie on $P_{1}$ and $P_{2}$ ?
(A) $\frac{x-1}{0}=\frac{y-1}{0}=\frac{z-1}{5}$
(C) $\frac{x}{-2}=\frac{y-4}{5}=\frac{z}{4}$
(B) $\frac{x-6}{-5}=\frac{y}{2}=\frac{z}{3}$
(D) $\frac{x}{1}=\frac{y-4}{-2}=\frac{z}{3}$

Ans. A,B,D

\begin{enumerate}
  \setcounter{enumi}{11}
  \item Let $S$ be the reflection of a point $Q$ with respect to the plane given by
\end{enumerate}

$$
\vec{r}=-(t+p) \hat{\imath}+t \hat{\jmath}+(1+p) \hat{k}
$$

where $t, p$ are real parameters and $\hat{l}, \hat{\jmath}, \hat{k}$ are the unit vectors along the three positive coordinate axes. If the position vectors of $Q$ and $S$ are $10 \hat{\imath}+15 \hat{\jmath}+20 \hat{k}$ and $\alpha \hat{\imath}+\beta \hat{\jmath}+\gamma \hat{k}$ respectively, then which of the following is/are TRUE?

Ans. $A, B, C$
(A) $3(\alpha+\beta)=-101$
(B) $3(\beta+\gamma)=-71$
(C) $3(\gamma+\alpha)=-86$
(D) $3(\alpha+\beta+\gamma)=-121$

\begin{enumerate}
  \setcounter{enumi}{12}
  \item Consider the parabola $y^{2}=4 x$. Let $S$ be the focus of the parabola. A pair of tangents drawn to the parabola from the point $P=(-2,1)$ meet the parabola at $P_{1}$ and $P_{2}$. Let $Q_{1}$ and $Q_{2}$ be points on the lines $S P_{1}$ and $S P_{2}$ respectively such that $P Q_{1}$ is perpendicular to $S P_{1}$ and $P Q_{2}$ is perpendicular to $S P_{2}$. Then, which of the following is/are TRUE?
\end{enumerate}

Ans. $\quad$ B,C,D
(A) $S Q_{1}=2$
(C) $\mathrm{PQ}_{1}=3$
(B) $Q_{1} Q_{2}=\frac{3 \sqrt{10}}{5}$
(D) $\mathrm{SQ}_{2}=1$

\begin{enumerate}
  \setcounter{enumi}{13}
  \item Let $|M|$ denote the determinant of a square matrix $M$. Let $g:\left[0, \frac{\pi}{2}\right] \rightarrow \mathbb{R}$ be the function defined by where
\end{enumerate}

$$
g(\theta)=\sqrt{f(\theta)-1}+\sqrt{f\left(\frac{\pi}{2}-\theta\right)-1}
$$

$f(\theta)=\frac{1}{2}\left|\begin{array}{ccc}1 & \sin \theta & 1 \\ -\sin \theta & 1 & \sin \theta \\ -1 & -\sin \theta & 1\end{array}\right|+\left|\begin{array}{ccc}\sin \pi & \cos \left(\theta+\frac{\pi}{4}\right) & \tan \left(\theta-\frac{\pi}{4}\right) \\ \sin \left(\theta-\frac{\pi}{4}\right) & -\cos \frac{\pi}{2} & \log _{e}\left(\frac{4}{\pi}\right) \\ \cot \left(\theta+\frac{\pi}{4}\right) & \log _{e}\left(\frac{\pi}{4}\right) & \tan \pi\end{array}\right|$

Let $\mathrm{p}(x)$ be a quadratic polynomial whose roots are the maximum and minimum values of the function $g(\theta)$, and $(2)=2-\sqrt{2}$. Then, which of the following is/are TRUE?

Ans. $A, C$
(A) $p\left(\frac{3+\sqrt{2}}{4}\right)<0$
(B) $p\left(\frac{1+3 \sqrt{2}}{4}\right)>0$
(C) $p\left(\frac{5 \sqrt{2}-1}{4}\right)>0$
(D) $p\left(\frac{5-\sqrt{2}}{4}\right)<0$

\section{MOTIOON JEE Advanced}
Question Paper

with Answer

\section{SECTION 3 (Maximum Marks: 12)}
\begin{itemize}
  \item This section contains FOUR (04) Matching List Sets.

  \item Each set has ONE Multiple Choice Question.

  \item Each set has TWO lists: List-I and List-II.

  \item List-I has Four entries (I), (II), (III) and (IV) and List-II has Five entries (P), (Q),(R), (S) and (T).

  \item FOUR options are given in each Multiple Choice Question based on List-I and List-II and ONLY ONE of these four options satisfies the condition asked in the Multiple Choice Question.

  \item Answer to each question will be evaluated according to the following marking scheme: Full Marks $\quad:+3$ ONLY if the option corresponding to the correct combination is chosen; Zero Marks : 0 If none of the options is chosen (i.e. the question is unanswered); Negative Marks : -1 In all other cases.

\end{itemize}

\begin{enumerate}
  \setcounter{enumi}{14}
  \item Consider the following lists:
\end{enumerate}

List - I

(I) $\left\{x \in\left[-\frac{2 \pi}{3}, \frac{2 \pi}{3}\right]: \cos x+\sin x=1\right\}$

(II) $\left\{x \in\left[-\frac{5 \pi}{18}, \frac{5 \pi}{18}\right]: \sqrt{3} \tan 3 x=1\right\}$

(III) $\left\{x \in\left[-\frac{6 \pi}{5}, \frac{6 \pi}{5}\right]: 2 \cos (2 x)=\sqrt{3}\right\}$

(IV) $\left\{x \in\left[-\frac{7 \pi}{4}, \frac{7 \pi}{4}\right]: \sin x-\cos x=1\right\}$

The correct option is:
(A) (I) $\rightarrow$ (P); (II) $\rightarrow$ (S); (III) $\rightarrow(\mathrm{P})$; (IV) $\rightarrow$ (S)
(C) (I) $\rightarrow(\mathrm{Q})$; (II) $\rightarrow$ (P); (III) $\rightarrow(\mathrm{T})$; (IV) $\rightarrow$ (S)
(B) (I) $\rightarrow$ (P); (II) $\rightarrow$ (P); (III) $\rightarrow$ (T); (IV) $\rightarrow$ (R)
(D) (I) $\rightarrow$ (Q); (II) $\rightarrow$ (S); (III) $\rightarrow$ (P); (IV) $\rightarrow$ (R)

\section{List - II}
(P) has two elements

(Q) has three elements

(R) has four elements

(S) has five elements

(T) has six elements

Ans. B

\begin{enumerate}
  \setcounter{enumi}{15}
  \item Two players, $P_{1}$ and $P_{2}$, play a game against each other. In every round of the game, each player rolls a fair die once, where the six faces of the die have six distinct numbers. Let $x$ and $y$ denote the readings on the die rolled by $P_{1}$ and $P_{2}$, respectively. If $x>y$, then $P_{1}$ scores 5 points and $P_{2}$ scores 0 point. If $x=y$, then each player scores 2 points. If $x<y$, then $P_{1}$ scores 0 point and $P_{2}$ scores 5 points. Let $X_{i}$ and $Y_{i}$ be the total scores of $P_{1}$ and $P_{2}$, respectively, after playing the $i^{\text {th }}$ round.
\end{enumerate}

List-I

(I) Probability of $\left(X_{2} \geq Y_{2}\right)$ is

(II) Probability of $\left(X_{2}>Y_{2}\right)$ is

(III) Probability of $\left(X_{3}=Y_{3}\right)$ is

(IV) Probability of $\left(X_{3}>Y_{3}\right)$ is

The correct option is:
(A) (I) $\rightarrow$ (Q); (II) $\rightarrow$ (R); (III) $\rightarrow$ (T); (IV) $\rightarrow$ (S)
(C) (I) $\rightarrow$ (P); (II) $\rightarrow$ (R); (III) $\rightarrow$ (Q); (IV) $\rightarrow$ (S)
(B) $(\mathrm{I}) \rightarrow(\mathrm{Q})$; (II) $\rightarrow(\mathrm{R})$; (III) $\rightarrow(\mathrm{T})$; (IV) $\rightarrow(\mathrm{T})$
(D) (I) $\rightarrow$ (P); (II) $\rightarrow$ (R); (III) $\rightarrow$ (Q); (IV) $\rightarrow$ (T)

List-II

(P) $\frac{3}{8}$

(Q) $\frac{11}{16}$

(R) $\frac{5}{16}$

(S) $\frac{355}{864}$

(T) $\frac{77}{432}$

Ans. A

\section{MOTIOON JEE Advanced}
Question Paper

with Answer

\begin{enumerate}
  \setcounter{enumi}{16}
  \item Let $p, q, r$ be nonzero real numbers that are, respectively, the $10^{\text {th }}, 100^{\text {th }}$ and $1000^{\text {th }}$ terms of a harmonic progression. Consider the system of linear equations
\end{enumerate}

$$
\begin{gathered}
x+y+z=1 \\
10 x+100 y+1000 z=0 \\
q r x+p r y+p q z=0 .
\end{gathered}
$$

List-I

(I) If $\frac{q}{r}=10$, then the system of linear equations has

(II) If $\frac{p}{r} \neq 100$, then the system of linear equations has

(III) If $\frac{p}{q} \neq 10$, then the system of linear equations has

(IV) If $\frac{p}{q}=10$, then the system of linear equations has

The correct option is:

Ans. B
(A) (I) $\rightarrow$ (T); (II) $\rightarrow$ (R); (III) $\rightarrow$ (S); (IV) $\rightarrow$ (T)
(B) (I) $\rightarrow$ (Q); (II) $\rightarrow$ (S); (III) $\rightarrow(\mathrm{S}) ;$ (IV) $\rightarrow(\mathrm{R})$
(D) (I) $\rightarrow(\mathrm{T})$; (II) $\rightarrow$ (S); (III) $\rightarrow(\mathrm{P})$; (IV) $\rightarrow(\mathrm{T})$
(C) (I) $\rightarrow$ (Q); (II) $\rightarrow$ (R); (III) $\rightarrow$ (P); (IV) $\rightarrow$ (R)
(B) (I) $\rightarrow$ (Q); (II) $\rightarrow$ (S); (III) $\rightarrow(\mathrm{S}) ;$ (IV) $\rightarrow(\mathrm{R})$
(D) (I) $\rightarrow(\mathrm{T})$; (II) $\rightarrow(\mathrm{S}) ;($ III) $\rightarrow(\mathrm{P}) ;$ (IV) $\rightarrow(\mathrm{T})$

(R) infinitely many solutions

(S) no solution

(T) at least one solution

\section{List-II}
(P) $x=0, y=\frac{10}{9}, z=-\frac{1}{9}$ as a solution

(Q) $x=\frac{10}{9}, y=-\frac{1}{9}, z=0$ as a solution

(S) no solution

\begin{enumerate}
  \setcounter{enumi}{17}
  \item Consider the ellipse
\end{enumerate}

$$
\frac{x^{2}}{4}+\frac{y^{2}}{3}=1
$$

Let $\mathrm{H}(\alpha, 0), 0<\alpha<2$, be a point. A straight line drawn through $H$ parallel to the $y$-axis crosses the ellipse and its auxiliary circle at points $E$ and $F$ respectively, in the first quadrant. The tangent to the ellipse at the point $E$ intersects the positive $x$-axis at a point $G$. Suppose the straight line joining $F$ and the origin makes an angle $\phi$ with the positive $x$-axis.

\section{List-I}
(I) If $\phi=\frac{\pi}{4}$, then the area of the triangle $F G H$ is

(II) If $\phi=\frac{\pi}{3}$, then the area of the triangle $F G H$ is

(III) If $\phi=\frac{\pi}{6}$, then the area of the triangle $F G H$ is

(IV) If $\phi=\frac{\pi}{12}$, then the area of the triangle $F G H$ is

List-II

(P) $\frac{(\sqrt{3}-1)^{4}}{8}$

(Q) 1

(R) $\frac{3}{4}$

(S) $\frac{1}{2 \sqrt{3}}$

(T) $\frac{3 \sqrt{3}}{2}$

The correct option is:

Ans. $\quad C$
(A) (I) $\rightarrow$ (R); (II) $\rightarrow$ (S); (III) $\rightarrow$ (Q); (IV) $\rightarrow$ (P)
(C) (I) $\rightarrow$ (Q); (II) $\rightarrow$ (T); (III) $\rightarrow$ (S); (IV) $\rightarrow$ (P)
(B) (I) $\rightarrow$ (R); (II) $\rightarrow$ (T); (III) $\rightarrow$ (S); (IV) $\rightarrow$ (P)
(D) (I) $\rightarrow$ (Q); (II) $\rightarrow$ (S); (III) $\rightarrow$ (Q); (IV) $\rightarrow$ (P)

JEE DROPPER BATCH

For Class 12th Pass Students


\end{document}